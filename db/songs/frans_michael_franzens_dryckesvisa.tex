\begin{song}
\songtitle{Frans Michael Franzéns dryckesvisa}
\firstline{När skämtet tar ordet vid vänskapens bord}
\alttitle{Bordsvisa}

\begin{songmeta}
Alternativ titel: Bordsvisa
Melodi: Frans Michael Franzéns dryckesvisa (trad.)
Text: Frans Michael Franzén
\end{songmeta}

\begin{songtext}
När skämtet tar ordet vid vänskapens bord
med fingret åt glasen som dofta,
så drick och var glad: på vår sorgliga jord
man gläder sig aldrig för ofta.

||: En blomma är glädjen, i dag slår hon ut,
i morgon förvissnar hon redan.
Just nu när du kan hav en lycklig minut
och tänk på den kommande sedan. :||

Vem drog ej en suck över tidernas lopp?
Dock sitt ej och dröm på kalaset!
Här lev i sekunden och hela ditt hopp
se fyllas och tömmas i glaset.

||: Här sörj ej för glaset: om fullt, så drick ut.
Om tomt, så försänd det att fyllas!
Och minns att det sköna och goda förut,
sen glädjen och nöjet, må hyllas. :||

Ty ägne vi först åt värdinnan en skål.
Vad vore vår fröjd utan henne?
Sen prise vi värden och särskilt hans bål.
Vad vore vårt mod utan denne?

||: Dem båda förene ett glas och en sång:
de själva så skönt sig förente.
Med druvorna myrten blev skapt på en gång:
vem ser ej vad himmelen mente? :||

För övrigt må värden ge alltid nytt skäl
till ständig omsättning av glasen
och visa, att rangen är nyttig likväl --
till skålarnas mängd på kalasen!

||: Men förr'n han är färdig med klang och harang,
vi skynda att självmant dricka
och helga ett glas, som är över all rang,
i tysthet - envar åt sin flicka! :||
\end{songtext}

\end{song}
