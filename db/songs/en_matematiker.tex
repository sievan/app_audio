\begin{song}
\songtitle{En Matematiker}

\begin{songmeta}
Melodi: En sockerbagare
Text: Mathias Lundgren
\end{songmeta}

\begin{songtext}
En matematiker här bor i staden,
han räknar matte mest hela dagen.
Han räknar primtal, stora och små,
han tycker om x upphöjt i två.

Och i hans fönster hänger skumma saker,
Där finns det sfärer, ibland kvadrater.
Och små ellipser och andra kägelsnitt
Låter det roligt så ta en titt.

Vår matematiker som bor i staden
han deriverar i högsta graden.
Han äter cous-cous (cos-kvadrat) och basmat(i)ris
Det verkar vettigt på någe vis.

Han skriver räkneböcker för studenter,
och han gör uppgifter till massa tentor.
Och är han snäller blir svaret två,
men är han stygger så blir det \begin{math}\rho\end{math}.
\end{songtext}

\begin{songnotes}
\begin{math}\rho\end{math} syftar på den grekiska bokstaven, vilket ibland
benämns som \textquotedblleft{}the plastic number\textquotedblright{} vilket är
den unika reella \\ lösningen på ekvationen $x^3 = x + 1$.
\begin{math}\rho\end{math} kallas ibland även för \textquotedblleft{}the silver
number\textquotedblright{} vilket används för \textquotedblleft{}the silver
ratio\textquotedblright{}: $1 + \sqrt{2}$.
\end{songnotes}
\end{song}
