\begin{song}
\songtitle{Kalmarevisan}
\firstline{Uti Kalmare stad}

\begin{songmeta}
Text: G S Kallstenius
\end{songmeta}

\begin{songtext}
||: Uti Kalmare stad,
ja där finns det ingen kvast :||
förrän lördagen.

Hej dick!
Hej dack!
Jag slog i,
och vi drack.
Hej dickom dickom dack!
Hej dickom dickom dack!
För uti Kalmare stad,
ja där finns det ingen kvast
förrän lördagen.

||: När som bonden kommer hem
kommer bondekvinnan ut :||
och är så stor i sin trut.

Hej dick\ldots

||: \textquotedblleft{}Var är pengarna du fått?\textquotedblright{}
\textquotedblleft{}Jo, dom har jag supit opp :||
uppå Kalmare slott.\textquotedblright{}

Hej dick\ldots
\newpage
\noindent||: \textquotedblleft{}Jag ska mäla dig an
för vår kronbefallningsman :||
och du ska få skam.\textquotedblright{}

Hej dick\ldots

||: \textquotedblleft{}Kronbefallningsmannen vår
satt på krogen igår :||
och var full som ett får.\textquotedblright{}

Hej dick\ldots
\end{songtext}

\begin{songnotes}
Första versraderna samt varannan refrängrad sjungs av en försångare.
Det finns även ett otal fler verser, som tillkommit vid olika tillfällen och platser.
\end{songnotes}

\end{song}
